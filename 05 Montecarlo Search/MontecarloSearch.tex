\documentclass[11pt,]{article}
\usepackage[left=1in,top=1in,right=1in,bottom=1in]{geometry}
\newcommand*{\authorfont}{\fontfamily{phv}\selectfont}
\usepackage[]{mathpazo}


  \usepackage[T1]{fontenc}
  \usepackage[utf8]{inputenc}




\usepackage{abstract}
\renewcommand{\abstractname}{}    % clear the title
\renewcommand{\absnamepos}{empty} % originally center

\renewenvironment{abstract}
 {{%
    \setlength{\leftmargin}{0mm}
    \setlength{\rightmargin}{\leftmargin}%
  }%
  \relax}
 {\endlist}

\makeatletter
\def\@maketitle{%
  \newpage
%  \null
%  \vskip 2em%
%  \begin{center}%
  \let \footnote \thanks
    {\fontsize{18}{20}\selectfont\raggedright  \setlength{\parindent}{0pt} \@title \par}%
}
%\fi
\makeatother




\setcounter{secnumdepth}{0}


\usepackage{graphicx,grffile}
\makeatletter
\def\maxwidth{\ifdim\Gin@nat@width>\linewidth\linewidth\else\Gin@nat@width\fi}
\def\maxheight{\ifdim\Gin@nat@height>\textheight\textheight\else\Gin@nat@height\fi}
\makeatother
% Scale images if necessary, so that they will not overflow the page
% margins by default, and it is still possible to overwrite the defaults
% using explicit options in \includegraphics[width, height, ...]{}
\setkeys{Gin}{width=\maxwidth,height=\maxheight,keepaspectratio}


\title{Programación Estadística: Búsqueda Montecarlo  }



\author{\Large Adrián Sosa\vspace{0.05in} \newline\normalsize\emph{}   \and \Large \vspace{0.05in} \newline\normalsize\emph{Universidad Veracruzana}  }



\date{}

\usepackage{titlesec}

\titleformat*{\section}{\normalsize\bfseries}
\titleformat*{\subsection}{\normalsize\itshape}
\titleformat*{\subsubsection}{\normalsize\itshape}
\titleformat*{\paragraph}{\normalsize\itshape}
\titleformat*{\subparagraph}{\normalsize\itshape}


\usepackage{natbib}
\bibliographystyle{plainnat}
\usepackage[strings]{underscore} % protect underscores in most circumstances



\newtheorem{hypothesis}{Hypothesis}
\usepackage{setspace}


% set default figure placement to htbp
\makeatletter
\def\fps@figure{htbp}
\makeatother

\usepackage{hyperref}

% move the hyperref stuff down here, after header-includes, to allow for - \usepackage{hyperref}

\makeatletter
\@ifpackageloaded{hyperref}{}{%
\ifxetex
  \PassOptionsToPackage{hyphens}{url}\usepackage[setpagesize=false, % page size defined by xetex
              unicode=false, % unicode breaks when used with xetex
              xetex]{hyperref}
\else
  \PassOptionsToPackage{hyphens}{url}\usepackage[draft,unicode=true]{hyperref}
\fi
}

\@ifpackageloaded{color}{
    \PassOptionsToPackage{usenames,dvipsnames}{color}
}{%
    \usepackage[usenames,dvipsnames]{color}
}
\makeatother
\hypersetup{breaklinks=true,
            bookmarks=true,
            pdfauthor={Adrián Sosa () and  (Universidad Veracruzana)},
            pdfkeywords = {},  
            pdftitle={Programación Estadística: Búsqueda Montecarlo},
            colorlinks=true,
            citecolor=blue,
            urlcolor=blue,
            linkcolor=magenta,
            pdfborder={0 0 0}}
\urlstyle{same}  % don't use monospace font for urls

% Add an option for endnotes. -----


% add tightlist ----------
\providecommand{\tightlist}{%
\setlength{\itemsep}{0pt}\setlength{\parskip}{0pt}}

% add some other packages ----------

% \usepackage{multicol}
% This should regulate where figures float
% See: https://tex.stackexchange.com/questions/2275/keeping-tables-figures-close-to-where-they-are-mentioned
\usepackage[section]{placeins}


\begin{document}
	
% \pagenumbering{arabic}% resets `page` counter to 1 
%
% \maketitle

{% \usefont{T1}{pnc}{m}{n}
\setlength{\parindent}{0pt}
\thispagestyle{plain}
{\fontsize{18}{20}\selectfont\raggedright 
\maketitle  % title \par  

}

{
   \vskip 13.5pt\relax \normalsize\fontsize{11}{12} 
\textbf{\authorfont Adrián Sosa} \hskip 15pt \emph{\small }   \par \textbf{\authorfont } \hskip 15pt \emph{\small Universidad Veracruzana}   
}

}






\vskip -8.5pt


 % removetitleabstract

\noindent  

\hypertarget{buxfasqueda-montecarlo}{%
\subsection{Búsqueda Montecarlo}\label{buxfasqueda-montecarlo}}

Monte Carlo es un metodo númerico muy versatil y facil de implementar,
se puede aplicar a problemas de N-dimensiones, en contraste con búsqueda
en malla, el método consiste en la elaboración de N puntos aleatorios
usando una distribución de probabilidad sobre el dominio del problema,
La complejidad del esfuerzo computacional es \(\Phi(N)\).

El árbol de búsqueda de Monte Carlo, es un algoritmo de búsqueda
heurístico para algunos tipos de procesos de toma de decisiones sobre
todo los que trabajan con juegos. Así como el método de operación el
enfoque de búsqueda de Monte Carlo se encuentra en el análisis de los
movimientos mas prometedores, ampliando el árbol de búsqueda, basado en
un muestreo aleatorio.

\hypertarget{procedimiento}{%
\subsubsection{Procedimiento}\label{procedimiento}}

\emph{Selección:} empezar desde la raíz \emph{R} y seleccionar nodos
hijos sucesivos hasta alcanzar un nodo hoja \emph{L}. La selección

\begin{figure}
\centering
\includegraphics{"/diagramas/uno.jpg"}
\caption{diagrama 1}
\end{figure}

\newpage

\hypertarget{codificaciuxf3n}{%
\subsubsection{Codificación}\label{codificaciuxf3n}}

\newpage
\singlespacing 
\end{document}