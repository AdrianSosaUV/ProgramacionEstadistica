\documentclass[11pt,]{article}
\usepackage[left=1in,top=1in,right=1in,bottom=1in]{geometry}
\newcommand*{\authorfont}{\fontfamily{phv}\selectfont}
\usepackage[]{mathpazo}


  \usepackage[T1]{fontenc}
  \usepackage[utf8]{inputenc}




\usepackage{abstract}
\renewcommand{\abstractname}{}    % clear the title
\renewcommand{\absnamepos}{empty} % originally center

\renewenvironment{abstract}
 {{%
    \setlength{\leftmargin}{0mm}
    \setlength{\rightmargin}{\leftmargin}%
  }%
  \relax}
 {\endlist}

\makeatletter
\def\@maketitle{%
  \newpage
%  \null
%  \vskip 2em%
%  \begin{center}%
  \let \footnote \thanks
    {\fontsize{18}{20}\selectfont\raggedright  \setlength{\parindent}{0pt} \@title \par}%
}
%\fi
\makeatother




\setcounter{secnumdepth}{0}

\usepackage{color}
\usepackage{fancyvrb}
\newcommand{\VerbBar}{|}
\newcommand{\VERB}{\Verb[commandchars=\\\{\}]}
\DefineVerbatimEnvironment{Highlighting}{Verbatim}{commandchars=\\\{\}}
% Add ',fontsize=\small' for more characters per line
\usepackage{framed}
\definecolor{shadecolor}{RGB}{248,248,248}
\newenvironment{Shaded}{\begin{snugshade}}{\end{snugshade}}
\newcommand{\AlertTok}[1]{\textcolor[rgb]{0.94,0.16,0.16}{#1}}
\newcommand{\AnnotationTok}[1]{\textcolor[rgb]{0.56,0.35,0.01}{\textbf{\textit{#1}}}}
\newcommand{\AttributeTok}[1]{\textcolor[rgb]{0.77,0.63,0.00}{#1}}
\newcommand{\BaseNTok}[1]{\textcolor[rgb]{0.00,0.00,0.81}{#1}}
\newcommand{\BuiltInTok}[1]{#1}
\newcommand{\CharTok}[1]{\textcolor[rgb]{0.31,0.60,0.02}{#1}}
\newcommand{\CommentTok}[1]{\textcolor[rgb]{0.56,0.35,0.01}{\textit{#1}}}
\newcommand{\CommentVarTok}[1]{\textcolor[rgb]{0.56,0.35,0.01}{\textbf{\textit{#1}}}}
\newcommand{\ConstantTok}[1]{\textcolor[rgb]{0.00,0.00,0.00}{#1}}
\newcommand{\ControlFlowTok}[1]{\textcolor[rgb]{0.13,0.29,0.53}{\textbf{#1}}}
\newcommand{\DataTypeTok}[1]{\textcolor[rgb]{0.13,0.29,0.53}{#1}}
\newcommand{\DecValTok}[1]{\textcolor[rgb]{0.00,0.00,0.81}{#1}}
\newcommand{\DocumentationTok}[1]{\textcolor[rgb]{0.56,0.35,0.01}{\textbf{\textit{#1}}}}
\newcommand{\ErrorTok}[1]{\textcolor[rgb]{0.64,0.00,0.00}{\textbf{#1}}}
\newcommand{\ExtensionTok}[1]{#1}
\newcommand{\FloatTok}[1]{\textcolor[rgb]{0.00,0.00,0.81}{#1}}
\newcommand{\FunctionTok}[1]{\textcolor[rgb]{0.00,0.00,0.00}{#1}}
\newcommand{\ImportTok}[1]{#1}
\newcommand{\InformationTok}[1]{\textcolor[rgb]{0.56,0.35,0.01}{\textbf{\textit{#1}}}}
\newcommand{\KeywordTok}[1]{\textcolor[rgb]{0.13,0.29,0.53}{\textbf{#1}}}
\newcommand{\NormalTok}[1]{#1}
\newcommand{\OperatorTok}[1]{\textcolor[rgb]{0.81,0.36,0.00}{\textbf{#1}}}
\newcommand{\OtherTok}[1]{\textcolor[rgb]{0.56,0.35,0.01}{#1}}
\newcommand{\PreprocessorTok}[1]{\textcolor[rgb]{0.56,0.35,0.01}{\textit{#1}}}
\newcommand{\RegionMarkerTok}[1]{#1}
\newcommand{\SpecialCharTok}[1]{\textcolor[rgb]{0.00,0.00,0.00}{#1}}
\newcommand{\SpecialStringTok}[1]{\textcolor[rgb]{0.31,0.60,0.02}{#1}}
\newcommand{\StringTok}[1]{\textcolor[rgb]{0.31,0.60,0.02}{#1}}
\newcommand{\VariableTok}[1]{\textcolor[rgb]{0.00,0.00,0.00}{#1}}
\newcommand{\VerbatimStringTok}[1]{\textcolor[rgb]{0.31,0.60,0.02}{#1}}
\newcommand{\WarningTok}[1]{\textcolor[rgb]{0.56,0.35,0.01}{\textbf{\textit{#1}}}}



\title{Programación Estadística: Búsqueda ciega  }



\author{\Large Adrián Sosa\vspace{0.05in} \newline\normalsize\emph{}   \and \Large \vspace{0.05in} \newline\normalsize\emph{Universidad Veracruzana}  }



\date{}

\usepackage{titlesec}

\titleformat*{\section}{\normalsize\bfseries}
\titleformat*{\subsection}{\normalsize\itshape}
\titleformat*{\subsubsection}{\normalsize\itshape}
\titleformat*{\paragraph}{\normalsize\itshape}
\titleformat*{\subparagraph}{\normalsize\itshape}


\usepackage{natbib}
\bibliographystyle{plainnat}
\usepackage[strings]{underscore} % protect underscores in most circumstances



\newtheorem{hypothesis}{Hypothesis}
\usepackage{setspace}


% set default figure placement to htbp
\makeatletter
\def\fps@figure{htbp}
\makeatother

\usepackage{hyperref}

% move the hyperref stuff down here, after header-includes, to allow for - \usepackage{hyperref}

\makeatletter
\@ifpackageloaded{hyperref}{}{%
\ifxetex
  \PassOptionsToPackage{hyphens}{url}\usepackage[setpagesize=false, % page size defined by xetex
              unicode=false, % unicode breaks when used with xetex
              xetex]{hyperref}
\else
  \PassOptionsToPackage{hyphens}{url}\usepackage[draft,unicode=true]{hyperref}
\fi
}

\@ifpackageloaded{color}{
    \PassOptionsToPackage{usenames,dvipsnames}{color}
}{%
    \usepackage[usenames,dvipsnames]{color}
}
\makeatother
\hypersetup{breaklinks=true,
            bookmarks=true,
            pdfauthor={Adrián Sosa () and  (Universidad Veracruzana)},
            pdfkeywords = {},  
            pdftitle={Programación Estadística: Búsqueda ciega},
            colorlinks=true,
            citecolor=blue,
            urlcolor=blue,
            linkcolor=magenta,
            pdfborder={0 0 0}}
\urlstyle{same}  % don't use monospace font for urls

% Add an option for endnotes. -----


% add tightlist ----------
\providecommand{\tightlist}{%
\setlength{\itemsep}{0pt}\setlength{\parskip}{0pt}}

% add some other packages ----------

% \usepackage{multicol}
% This should regulate where figures float
% See: https://tex.stackexchange.com/questions/2275/keeping-tables-figures-close-to-where-they-are-mentioned
\usepackage[section]{placeins}


\begin{document}
	
% \pagenumbering{arabic}% resets `page` counter to 1 
%
% \maketitle

{% \usefont{T1}{pnc}{m}{n}
\setlength{\parindent}{0pt}
\thispagestyle{plain}
{\fontsize{18}{20}\selectfont\raggedright 
\maketitle  % title \par  

}

{
   \vskip 13.5pt\relax \normalsize\fontsize{11}{12} 
\textbf{\authorfont Adrián Sosa} \hskip 15pt \emph{\small }   \par \textbf{\authorfont } \hskip 15pt \emph{\small Universidad Veracruzana}   
}

}






\vskip -8.5pt


 % removetitleabstract

\noindent \doublespacing 

\hypertarget{buxfasqueda-ciega}{%
\section{Búsqueda ciega}\label{buxfasqueda-ciega}}

La búsqueda ciega completa asume el agotamiento de todas las
alternativas, donde cualquier búsqueda previa no afecta la forma en que
se prueban las siguientes soluciones. Dado que Se prueba el espacio de
búsqueda completo, siempre se encuentra la solución óptima. La búsqueda
ciega solo es aplicable a espacios de búsqueda discretos y es fácil de
codificar de dos formas.

Primero, configurando el espacio de búsqueda completo en una matriz y
luego probando secuencialmente cada fila (solución) de esta matriz.

En segundo lugar, de forma recursiva, configurando el espacio de
búsqueda como un árbol, donde cada rama denota un valor posible para una
variable dada y todas las soluciones aparecen en las hojas (al mismo
nivel).

Dos ejemplos bastante conocidos delos métodos ciegos basados en
estructuras de árbol son los algoritmos de profundidad y de amplitud.

El primero comienza en la raíz del árbol y atraviesa cada rama tanto
como esposible, antes de dar marcha atrás. El segundo también comienza
en la raíz pero busca en un nivel base, buscando primero todos los nodos
sucesivos de la raíz y luego el siguiente sucesivo a los nodos, etc.

La principal desventaja de la búsqueda ciega pura es que no es factible
cuando el espacio de búsqueda es continuo o demasiado grande, una
situación que a menudo ocurre con las tareas del mundo real.

Esta sección presenta dos funciones de búsqueda ciega: fsearch y
dfsearch. La primera es una función más simple que requiere que el
espacio de búsqueda se defina explícitamente en una matriz en el formato
soluciones D (búsqueda de argumento), mientras que la última realiza una
implementación recursiva de la búsqueda en profundidad y requiere la
definición de los valores de dominio para cada variable a optimizar
(dominio de argumento). Ambas funciones reciben como argumentos la
función de evaluación (FUN), el tipo de optimización (tipo, un carácter
con ``min'' o ``max'') y argumentos adicionales, (denotados por \ldots{}
y que podría ser utilizado por la función de evaluación FUN).

Donde dfsearch es una función recursiva que prueba si el nodo del árbol
es una salida, calcula la función de evaluación para la solución
respectiva, de lo contrario atraviesa las sub ramas del nodo. Esta
función requiere algunas variables de estado de memoria (l, b, x y msol)
que se cambian cada vez que se ejecuta una nueva llamada recursiva. El
dominio de valores se almacena en una lista de vectores de longitud D,
ya que los elementos de este vector pueden tener diferentes longitudes,
de acuerdo con sus valores de dominio.

\newpage

\hypertarget{codificaciuxf3n}{%
\subsubsection{Codificación}\label{codificaciuxf3n}}

La codificación de dichas funciones podemos observala acontinuación:

\begin{Shaded}
\begin{Highlighting}[]
\NormalTok{fsearch}
\end{Highlighting}
\end{Shaded}

\begin{verbatim}
## function (sol, Fx, type = "min", ...) 
## {
##     x <- apply(sol, 1, Fx, ...)
##     ib <- switch(type, min = which.min(x), max = which.max(x))
##     return(list(index = ib, sol = sol[ib], eval = x[ib]))
## }
\end{verbatim}

\begin{Shaded}
\begin{Highlighting}[]
\NormalTok{dfsearch }
\end{Highlighting}
\end{Shaded}

\begin{verbatim}
## function (domino, Fx, l = 1, b = 1, type = "min", D = length(dominio), 
##     x = rep(NA, D), msol = switch(type, min = list(sol = NULL, 
##         eval = Inf), max = list(sol = NULL, eval = -Inf)), ...) 
## {
##     if ((l - 1) == D) {
##         f <- Fx(x, ...)
##         fb <- msol$eval
##         ib <- switch(type, min = which.min(c(fb, f)), max = which.max(c(fb, 
##             f)))
##         if (ib == 1) 
##             return(msol)
##         else return(list(index = ib, sol = x, eval = f, bsol = x[ib], 
##             beval = f[ib]))
##     }
##     else {
##         for (j in 1:length(dominio[[l]])) {
##             x[l] <- dominio[[l]][j]
##             msol <- dfsearch(dominio, Fx, l + 1, j, type, D = D, 
##                 x = x, msol = msol, ...)
##         }
##         return(msol)
##     }
## }
\end{verbatim}

Se crea un espacio de búsqueda con valores entre 0 y 9 en una matriz de
5 *7 y se define una función:

\begin{Shaded}
\begin{Highlighting}[]
\CommentTok{# función a evaluar}
\NormalTok{fx1}
\end{Highlighting}
\end{Shaded}

\begin{verbatim}
## function (x) 
## {
##     return(x^2 + 3 * x - 2)
## }
\end{verbatim}

\begin{Shaded}
\begin{Highlighting}[]
\NormalTok{m <-}\StringTok{ }\DecValTok{5}  \CommentTok{# número de variables}
\NormalTok{n <-}\StringTok{ }\DecValTok{5}  \CommentTok{# tamaño del espacio}

\CommentTok{# definimos un espacio de busqueda}
\NormalTok{dominio <-}\StringTok{ }\KeywordTok{matrix}\NormalTok{(}\DataTypeTok{data=}\KeywordTok{sample}\NormalTok{(}\DecValTok{0}\OperatorTok{:}\DecValTok{9}\NormalTok{, m}\OperatorTok{*}\NormalTok{n,}\OtherTok{TRUE}\NormalTok{),n,m)}
\CommentTok{# mostramos el espacio de búsqueda}
\NormalTok{dominio}
\end{Highlighting}
\end{Shaded}

\begin{verbatim}
##      [,1] [,2] [,3] [,4] [,5]
## [1,]    9    3    3    7    5
## [2,]    8    4    6    7    7
## [3,]    3    2    3    8    0
## [4,]    5    7    8    1    7
## [5,]    7    4    2    2    7
\end{verbatim}

El siguiente código prueba el funcionamiento del método de búsqueda
ciega en amplitud:

\begin{Shaded}
\begin{Highlighting}[]
\CommentTok{# búsqueda ciega en amplitud}
\NormalTok{min <-}\StringTok{ }\KeywordTok{fsearch}\NormalTok{(dominio,}\DataTypeTok{Fx=}\NormalTok{fx1,}\DataTypeTok{type=}\StringTok{"min"}\NormalTok{)}
\NormalTok{max <-}\StringTok{ }\KeywordTok{fsearch}\NormalTok{(dominio,}\DataTypeTok{Fx=}\NormalTok{fx1,}\DataTypeTok{type=}\StringTok{"max"}\NormalTok{)}
\end{Highlighting}
\end{Shaded}

\begin{verbatim}
## Solución mínima:  2 
##  Evaluación:  -2
\end{verbatim}

\begin{verbatim}
## Solución máxima:  9 
##  Evaluación:  106
\end{verbatim}

El siguiente código prueba el funcionamiento de la fúncion de búsqueda
ciega en profundidad:

\begin{Shaded}
\begin{Highlighting}[]
\CommentTok{# búsqueda ciega en profundidad}
\NormalTok{min <-}\StringTok{ }\KeywordTok{dfsearch}\NormalTok{(dominio,}\DataTypeTok{Fx=}\NormalTok{fx1,}\DataTypeTok{type=}\StringTok{"min"}\NormalTok{)}
\NormalTok{max <-}\StringTok{ }\KeywordTok{dfsearch}\NormalTok{(dominio,}\DataTypeTok{Fx=}\NormalTok{fx1,}\DataTypeTok{type=}\StringTok{"max"}\NormalTok{)}
\end{Highlighting}
\end{Shaded}

\begin{verbatim}
## Solución mínima:  7 
##  Evaluación:  68
\end{verbatim}

\begin{verbatim}
## Solución máxima:  8 
##  Evaluación:  86
\end{verbatim}

\newpage
\singlespacing 
\end{document}